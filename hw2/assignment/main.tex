%%%%%%%%%%%%%%%%%%%%%%%%%%%%%%%%%%%%%%%%%
% fphw Assignment
% LaTeX Template
% Version 1.0 (27/04/2019)
%
% This template originates from:
% https://www.LaTeXTemplates.com
%
% Authors:
% Class by Felipe Portales-Oliva (f.portales.oliva@gmail.com) with template 
% content and modifications by Vel (vel@LaTeXTemplates.com)
%
% Template (this file) License:
% CC BY-NC-SA 3.0 (http://creativecommons.org/licenses/by-nc-sa/3.0/)
%
%%%%%%%%%%%%%%%%%%%%%%%%%%%%%%%%%%%%%%%%%

%----------------------------------------------------------------------------------------
%	PACKAGES AND OTHER DOCUMENT CONFIGURATIONS
%----------------------------------------------------------------------------------------

\documentclass[
	12pt, % Default font size, values between 10pt-12pt are allowed
	%letterpaper, % Uncomment for US letter paper size
	%spanish, % Uncomment for Spanish
]{fphw}

\usepackage{lmodern}
\usepackage{amsfonts}
\usepackage{epstopdf}
%\usepackage{matlab}
% Template-specific packages
\usepackage[utf8]{inputenc} % Required for inputting international characters
\usepackage[T1]{fontenc} % Output font encoding for international character
\usepackage[italian]{babel}
\usepackage{mathpazo} % Use the Palatino font

\usepackage{graphicx} % Required for including images

\usepackage{booktabs} % Required for better horizontal rules in tables

\usepackage{listings} % Required for insertion of code

\usepackage{enumerate} % To modify the enumerate environment

\usepackage{hyperref}

\usepackage{amsmath}

\usepackage{amssymb}

\usepackage{amsthm}

\usepackage{color}

\usepackage{optidef}

\usepackage{tikzit}

\usepackage{tikz,tkz-graph}
\usetikzlibrary{automata, calc, positioning, quotes}


\usepackage{mathrsfs}

\usepackage{enumitem}
\usepackage{float}

\sloppy
\epstopdfsetup{outdir=./}
\graphicspath{ {./figures/} }

\hypersetup{
    colorlinks=true,
    linkcolor=blue,
    filecolor=magenta,      
    urlcolor=cyan,
    }

\definecolor{backcolour}{rgb}{0.98, 0.98, 0.95}
\definecolor{codegreen}{rgb}{0,0.6,0}
\definecolor{codeorange}{rgb}{1,0.58,0}

\lstdefinestyle{mystyle}{
	backgroundcolor=\color{backcolour},   
    commentstyle=\slshape\color{codegreen},
    numberstyle=\tiny,
    keywordstyle=\bfseries\color{blue},
    basicstyle=\ttfamily\small,
    breakatwhitespace=false,         
    breaklines=true,                 
    captionpos=b,                    
    keepspaces=true,                 
    %numbers=left,                    
    %numbersep=5pt,                  
    showspaces=false,                
    showstringspaces=false,
    showtabs=false,                  
    tabsize=2,
    frame = single,
    captionpos = t
}
\lstset{style=mystyle}

\DeclareMathOperator{\Ev}{\mathbb{E}}% expected value
\DeclareMathOperator{\Pro}{\mathbb{P}}% probability

\renewcommand{\labelenumi}{\textit{(\alph{enumi})}}

%----------------------------------------------------------------------------------------
%	ASSIGNMENT INFORMATION
%----------------------------------------------------------------------------------------

\title{Homework 2} % Assignment title

\author{Andrea Ruglioni} % Student name

\group{Giulia Monchietto}

\date{20 Febbraio 2022} % Due date

\institute{Politecnico di Torino} % Institute or school name

\class{Processi stocastici} % Course or class name

\professor{Barbara Trivellato} % Professor or teacher in charge of the assignment

\newtheorem*{claim}{Claim}

%----------------------------------------------------------------------------------------

\begin{document}


\maketitle % Output the assignment title, created automatically using the information in the custom commands above

%----------------------------------------------------------------------------------------
%	ASSIGNMENT CONTENT
%----------------------------------------------------------------------------------------

%\section*{Suddivisione del lavoro}
%Ciascun membro del gruppo ha svolto tutti gli esercizi e ha revisionato l'intero homework.
%La stesura in Latex del lavoro è stata suddivisa tra i partecipanti, in particolare mi sono occupato della scrittura dell'esercizio 2 insieme a Giulia Monchietto.

\section*{Esercizio}
\begin{problem}
	Si consideri il problema della rovina del giocatore nel caso di scommesse non equilibrate,
	quando cioè il gioco sia dato da $X_0 = 0, X_n = Y_1 + \dots + Y_n$, con $Y_i$ successione di variabli
	aleatorie indipendenti di ugual legge $\Pro(Y_i = 1) = p, \Pro(Y_i = -1) = q = 1 - p$ e $p \neq q$. Per
	fissare le idee, supponiamo $p > q > 0$. Per questo gioco, sia $\tau$ il suo primo tempo d'ingresso
	nell'insieme $\{a, -b\}$, con $a$, $b$ interi positivi. Intuitivamente, si tratta di scommettere al gioco
	fino all'esaurimento delle disponibilità finanziarie di uno dei due giocatori, che si suppone
	possiedano l'uno $a$ e l'altro $b$ monete.

	\smallskip
\end{problem}

\section*{Domanda 1}
\begin{problem}
	Giustificare perché il tempo d'arresto $\tau$ è finito quasi certamente.
	\smallskip
\end{problem}

\subsection*{Soluzione}



\section*{Domanda 2}
\begin{problem}
	Dimostrare che il processo $Z_0 = 1, Z_n = \alpha^{X_n}$, con $\alpha = q/p(< 1)$ è una martingala
	rispetto alla filtrazione $(\mathcal{F}_n)_n$, con $\mathcal{F}_n = \sigma\{Y_1, \dots , Y_n\}$.
	\smallskip
\end{problem}

\subsection*{Soluzione}



\section*{Domanda 3}
\begin{problem}
	Calcolare $m_a = \Pro(X_{\tau} = a)$.
	\smallskip
\end{problem}

\subsection*{Soluzione}
Dal punto precedente sappiamo che $Z_n = \alpha^{X_n}$ è una Martingala non negativa, di conseguenza il processo arrestato $Z_n^{\tau}$ possiederà anch'esso le proprietà di Martingala non negativa.\\
In particolare $Z_n^{\tau} \leq \max \{\alpha^a, \alpha^{-b}\}$, le ipotesi del Lemma di Doob sono quindi soddisfatte e vale:
\begin{equation*}
\mathbb{E}[Z_n^{\tau}] = \mathbb{E}[Z_0^{\tau}]  = \mathbb{E}[Z_0] = 1
\end{equation*}
Dato che $Z^{\tau} \in \{ \alpha^a, \alpha^{-b} \}$, ricordando che $\mathbb{P}(X_{\tau} = a) + \mathbb{P}(X_{\tau} = -b) = 1$ , si deduce che:
\begin{equation*}
\mathbb{E}[Z_n^{\tau}] = \alpha^a m_a + \alpha^{-b} (1-m_a) = 1
\end{equation*}
Di conseguenza ricaviamo la seguente soluzione:
\begin{equation*}
m_a = \frac{1-\alpha^{-b}}{\alpha^a-\alpha^{-b}}
\end{equation*}.

\section*{Domanda 4}
\begin{problem}
	Calcolare $\Ev(\tau)$.
	\smallskip
\end{problem}

\subsection*{Soluzione}
Per prima cosa osserviamo che
\begin{equation*}
\mathbb{E}[Y_i] = 2p-1
\end{equation*}
Segue che il processo definito come:
\begin{equation*}
M_n = \sum_{k = 1}^n (Y_k - (2p-1))  = X_n - n(2p-1)
\end{equation*}
è anch'esso una martingala.
Notiamo inoltre che $|M_n - M_{n-1}| \leq 1 + |2p-1|$, perciò per il lemma di Doob le medie della martingala sono costanti.
Considerando la martingala arrestata con $\tau$, deduciamo:
\begin{equation*}
\mathbb{E}[L_{\tau}] = \mathbb{E}[X_{\tau} - \tau(2p-1)] = \mathbb{E}[L_{0}] = 0
\end{equation*}
Sfruttando la linearità del valore atteso otteniamo che:
\begin{equation*}
\mathbb{E}[\tau] = \frac{1}{2p-1} \mathbb{E}[X_{\tau}]
\end{equation*}
Sostituendo l'espressione di $m_a$ ottenuta sopra, otteniamo infine l'espressione del valore atteso di $\tau$ in funzione di $\alpha$.
\begin{equation*}
\mathbb{E}[\tau] = \frac{a}{2p-1} \frac{1-\alpha^{-b}}{\alpha^{a}-\alpha^{-b}} - \frac{b}{2p-1} \frac{\alpha^{a}-1}{\alpha^{a}-\alpha^{-b}}
\end{equation*}.


\end{document}

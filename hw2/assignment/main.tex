%%%%%%%%%%%%%%%%%%%%%%%%%%%%%%%%%%%%%%%%%
% fphw Assignment
% LaTeX Template
% Version 1.0 (27/04/2019)
%
% This template originates from:
% https://www.LaTeXTemplates.com
%
% Authors:
% Class by Felipe Portales-Oliva (f.portales.oliva@gmail.com) with template 
% content and modifications by Vel (vel@LaTeXTemplates.com)
%
% Template (this file) License:
% CC BY-NC-SA 3.0 (http://creativecommons.org/licenses/by-nc-sa/3.0/)
%
%%%%%%%%%%%%%%%%%%%%%%%%%%%%%%%%%%%%%%%%%

%----------------------------------------------------------------------------------------
%	PACKAGES AND OTHER DOCUMENT CONFIGURATIONS
%----------------------------------------------------------------------------------------

\documentclass[
	12pt, % Default font size, values between 10pt-12pt are allowed
	%letterpaper, % Uncomment for US letter paper size
	%spanish, % Uncomment for Spanish
]{fphw}

\usepackage{lmodern}
\usepackage{amsfonts}
\usepackage{epstopdf}
%\usepackage{matlab}
% Template-specific packages
\usepackage[utf8]{inputenc} % Required for inputting international characters
\usepackage[T1]{fontenc} % Output font encoding for international character
\usepackage[italian]{babel}
\usepackage{mathpazo} % Use the Palatino font

\usepackage{graphicx} % Required for including images

\usepackage{booktabs} % Required for better horizontal rules in tables

\usepackage{listings} % Required for insertion of code

\usepackage{enumerate} % To modify the enumerate environment

\usepackage{hyperref}

\usepackage{amsmath}

\usepackage{amssymb}

\usepackage{amsthm}

\usepackage{color}

\usepackage{optidef}

\usepackage{tikzit}

\usepackage{tikz,tkz-graph}
\usetikzlibrary{automata, calc, positioning, quotes}


\usepackage{mathrsfs}

\usepackage{enumitem}
\usepackage{float}

\sloppy
\epstopdfsetup{outdir=./}
\graphicspath{ {./figures/} }

\hypersetup{
    colorlinks=true,
    linkcolor=blue,
    filecolor=magenta,      
    urlcolor=cyan,
    }

\definecolor{backcolour}{rgb}{0.98, 0.98, 0.95}
\definecolor{codegreen}{rgb}{0,0.6,0}
\definecolor{codeorange}{rgb}{1,0.58,0}

\lstdefinestyle{mystyle}{
	backgroundcolor=\color{backcolour},   
    commentstyle=\slshape\color{codegreen},
    numberstyle=\tiny,
    keywordstyle=\bfseries\color{blue},
    basicstyle=\ttfamily\small,
    breakatwhitespace=false,         
    breaklines=true,                 
    captionpos=b,                    
    keepspaces=true,                 
    %numbers=left,                    
    %numbersep=5pt,                  
    showspaces=false,                
    showstringspaces=false,
    showtabs=false,                  
    tabsize=2,
    frame = single,
    captionpos = t
}
\lstset{style=mystyle}

\DeclareMathOperator{\Ev}{\mathbb{E}}% expected value
\DeclareMathOperator{\Pro}{\mathbb{P}}% probability
\DeclareMathOperator{\F}{\mathcal{F}}% filtrazione

%----------------------------------------------------------------------------------------
%	ASSIGNMENT INFORMATION
%----------------------------------------------------------------------------------------

\title{Homework 2} % Assignment title

\author{Andrea Ruglioni} % Student name

\group{Giulia Monchietto}

\date{20 Febbraio 2022} % Due date

\institute{Politecnico di Torino} % Institute or school name

\class{Processi stocastici} % Course or class name

\professor{Barbara Trivellato} % Professor or teacher in charge of the assignment

\newtheorem*{claim}{Claim}

%----------------------------------------------------------------------------------------

\begin{document}


\maketitle % Output the assignment title, created automatically using the information in the custom commands above

%----------------------------------------------------------------------------------------
%	ASSIGNMENT CONTENT
%----------------------------------------------------------------------------------------

%\section*{Suddivisione del lavoro}
%Ciascun membro del gruppo ha svolto tutti gli esercizi e ha revisionato l'intero homework.
%La stesura in Latex del lavoro è stata suddivisa tra i partecipanti, in particolare mi sono occupato della scrittura dell'esercizio 2 insieme a Giulia Monchietto.

\section*{Esercizio}
\begin{problem}
	Si consideri il problema della rovina del giocatore nel caso di scommesse non equilibrate,
	quando cioè il gioco sia dato da $X_0 = 0, X_n = Y_1 + \dots + Y_n$, con $Y_i$ successione di variabli
	aleatorie indipendenti di ugual legge $\Pro(Y_i = 1) = p, \Pro(Y_i = -1) = q = 1 - p$ e $p \neq q$. Per
	fissare le idee, supponiamo $p > q > 0$. Per questo gioco, sia $\tau$ il suo primo tempo d'ingresso
	nell'insieme $\{a, -b\}$, con $a$, $b$ interi positivi. Intuitivamente, si tratta di scommettere al gioco
	fino all'esaurimento delle disponibilità finanziarie di uno dei due giocatori, che si suppone
	possiedano l'uno $a$ e l'altro $b$ monete.

	\smallskip
\end{problem}

\section*{Domanda 1}
\begin{problem}
	Giustificare perché il tempo d'arresto $\tau$ è finito quasi certamente.
	\smallskip
\end{problem}

\subsection*{Soluzione}

Notiamo che le variabili aleatorie $Y_1, \dots, Y_n$ sono indipendenti identicamente distribuite ed appartenenti ad $L^2$.
Infatti, $\forall i \in \mathbb{N}$
\begin{equation*}
    \begin{gathered}
        \Ev[Y_i] = p - q = 2p - 1,\\
        \Ev[Y_i^2] = p + q = 1.
    \end{gathered}
\end{equation*}
Dunque, dalla legge forte dei grandi numeri segue che
\begin{equation*}
    \frac{\sum_{i=1}^n Y_i}{n} = \frac{X_n}{n} \xrightarrow[n \to \infty]{\text{q.c.}} 2p - 1.
\end{equation*}
Quindi, per definizione di limite, vale quasi certamente che, $\forall \epsilon > 0, \forall n \geq N \in \mathbb{N}$
\begin{equation*}
    \left| \frac{X_n}{n} - (2p-1) \right| < \epsilon.
\end{equation*}
In particolare, è valida la relazione
\begin{equation*}
    X_n > n(2p-1) - n \epsilon = n(2p-1-\epsilon).
\end{equation*}
Ricordando che $p > 1/2$, prendiamo $\epsilon = (2p-1)/2 > 0$, da cui si ottiene
\begin{equation*}
    X_n > n(p - \frac{1}{2}) \xrightarrow[n \to \infty]{} + \infty.
\end{equation*}
In conclusione, $\tau$ è finito quasi certamente perché, $\forall a \in \mathbb{N}, \exists \text{ q.c. } n^* \in \mathbb{N}$, tale per cui $X_{n^*} \geq a$.

\section*{Domanda 2}
\begin{problem}
	Dimostrare che il processo $Z_0 = 1, Z_n = \alpha^{X_n}$, con $\alpha = q/p(< 1)$ è una martingala
	rispetto alla filtrazione $(\mathcal{F}_n)_n$, con $\mathcal{F}_n = \sigma\{Y_1, \dots , Y_n\}$.
	\smallskip
\end{problem}

\subsection*{Soluzione}

Dimostriamo che sono soddisfatte le tre proprietà caratterizzanti della martingala.
\begin{enumerate}
    \item $Z_n \sim \mathcal{F}_n$. Infatti
        \begin{equation*}
            Z_n = \alpha^{X_n} = \alpha^{\sum_{i=1}^n Y_i},
        \end{equation*}
        quindi è funzione delle sole v.a. $Y_1, \dots, Y_n$ ed è, di conseguenza, misurabile rispetto alla filtrazione $\mathcal{F}_n = \sigma\{Y_1, \dots , Y_n\}$.
    \item $Z_n \in L^1$. Infatti
        \begin{equation*}
            \Ev[|Z_n|] = \Ev[Z_n] \leq \Ev[\alpha^{-n}] = \alpha^{-n} < \infty,
        \end{equation*}
        dove abbiamo utilizzato il fatto che $Z_n > 0$, che $Y_i >= -1 $ $\forall i \in \mathbb{N}$ e che l'esponenziale con $\alpha < 1$ è decrescente.
    \item Verifichiamo ora la validità della proprietà di martingala
        \begin{equation*}
            \begin{aligned}
                \Ev[Z_{n+1} | \F_n] &= \Ev[\alpha^{\sum_{i=1}^{n+1} Y_i} | \F_n]\\
                &= \Ev[Z_n \alpha^{Y_{n+1}} | \F_n]\\
                &= Z_n \Ev[\alpha^{Y_{n+1}} | \F_n]\\
                &= Z_n \Ev[\alpha^{Y_{n+1}}] \\
                &= Z_n.
            \end{aligned}
        \end{equation*}
        La terza e la quarta uguaglianza sono dovute alle proprietà del valore atteso condizionato, poichè $Z_n$ è $\F_n$-misurabile ed $Y_{n+1}$ è indipendente da $\F_n = \sigma\{Y_1, \dots , Y_n\}$.
        Infine, l'ultima uguaglianza è data da
        \begin{equation*}
            \Ev[\alpha^{Y_{n+1}}] = p \alpha + q \alpha^{-1} = p+q = 1.
        \end{equation*}
\end{enumerate}

\newpage
\section*{Domanda 3}
\begin{problem}
	Calcolare $m_a = \Pro(X_{\tau} = a)$.
	\smallskip
\end{problem}

\subsection*{Soluzione}
Dal punto precedente sappiamo che $Z_n = \alpha^{X_n}$ è una martingala, di conseguenza anche il processo arrestato $Z_n^{\tau}$ è una martingala.
Inoltre, questa è chiusa in quanto è dominata il $L^1$. Infatti, ricordando che $\alpha < 1$, si ha che
\begin{equation*}
    |Z_n^{\tau}| = Z_n^{\tau} \leq \alpha^{-b} \in L^1 \qquad \forall n \in \mathbb{N}.
\end{equation*}
Segue che è possibile applicare il teorema di arresto di Doob, il quale afferma che dati $\sigma \leq \tau$ tempi di arresto finiti quasi certamente, vale
\begin{equation*}
    \Ev[Z_{\tau} | \F_{\sigma}] = Z_{\sigma},
\end{equation*}
o equivalentemente, effettuando ad ambo i lati il valore atteso
\begin{equation*}
    \Ev[Z_{\tau}] = \Ev[Z_{\sigma}].
\end{equation*}
Ora, prendendo $\sigma = 0$, otteniamo l'uguaglianza $\Ev[Z_{\tau}] = \Ev[Z_0] = 1$.
Dato che $Z_{\tau} \in \{ \alpha^a, \alpha^{-b} \}$, e ricordando che $\mathbb{P}(X_{\tau} = a) + \mathbb{P}(X_{\tau} = -b) = m_a + m_{-b} = 1$ , si deduce che:
\begin{equation*}
    \begin{aligned}
        1 &= \Ev[Z_{\tau}]\\
        &= m_a \alpha^a + m_{-b} \alpha^{-b}\\
        &= m_a \alpha^a + (1 - m_a) \alpha^{-b}.
    \end{aligned}
\end{equation*}
Dalla quale si ricava il valore cercato
\begin{equation} \label{eq:ma}
    m_a = \frac{1-\alpha^{-b}}{\alpha^a-\alpha^{-b}}.
\end{equation}

\newpage
\section*{Domanda 4}
\begin{problem}
	Calcolare $\Ev(\tau)$.
	\smallskip
\end{problem}
\subsection*{Soluzione}
Ricordiamo che $\Ev[Y_i] = 2p-1,$ $\forall i \in \mathbb{N}$, da cui segue che il processo definito come
\begin{equation*}
    M_n = \sum_{i = 1}^n (Y_i - (2p-1))  = X_n - n(2p-1), \qquad M_0 = 0,
\end{equation*}
è una martingala, poichè somma di variabili aleatorie indipendenti e centrate.
Dunque, anche $(M_n^{\tau})_{n \in \mathbb{N}} = (M_{n \wedge \tau})_{n \in \mathbb{N}}$ è una martingala, ed avrà quindi media costante
\begin{equation*}
    \Ev[M_n^{\tau}] = \Ev[M_0^{\tau}] = \Ev[M_0] = 0.
\end{equation*}
Si ottiene quindi che
\begin{equation*}
    0 = \Ev[M_{n \wedge \tau}] = \Ev[X_{n \wedge \tau}] - \Ev[n \wedge \tau](2p-1),
\end{equation*}
o equivalentemente
\begin{equation*}
    \Ev[X_{n \wedge \tau}] = \Ev[n \wedge \tau](2p-1).
\end{equation*}
Calcoliamo il limite di ambo i lati per $n \to \infty$.
A sinistra si ottiene che
\begin{equation*}
    \lim_{n \to \infty} \Ev[X_{n \wedge \tau}] = \Ev[X_{\tau}],
\end{equation*}
perché $\tau$ è finito q.c. e per il teorema di convergenza dominata, infatti
\begin{equation*}
    |X_{n \wedge \tau}| \leq \max(a, b) \in L^1, \qquad \forall n \in \mathbb{N}.
\end{equation*}
Invece, a destra otteniamo
\begin{equation*}
    \lim_{n \to \infty} \Ev[n \wedge \tau](2p-1) = \Ev[\tau](2p-1),
\end{equation*}
ancora per il fatto che $\tau$ è finito q.c e grazie al teorema di Beppo Levi (convergenza monotona), in quanto la successione $(n \wedge \tau)_{n \in \mathbb{N}}$ è non decrescente.
In conclusione si ricava che
\begin{equation*}
        \Ev[\tau] = \frac{\Ev[X_{\tau}]}{2p - 1} = \frac{a m_a + b (1 - m_a)}{2p - 1},
\end{equation*}
dove $m_a$ è dato dall'equazione (\ref{eq:ma}) precedentemente ottenuta.


\end{document}